
\documentclass[11pt, oneside]{article}
\usepackage{geometry} 
\geometry{a4paper} 
%\usepackage[parfill]{parskip}  % Activate to begin paragraphs with an empty line rather than an indent
\usepackage{graphicx}			% Use pdf, png, jpg, or eps§ with pdflatex; use eps in DVI mode
								% TeX will automatically convert eps --> pdf in pdflatex		
\usepackage{amssymb}

\pagenumbering{gobble}

%SetFonts

%SetFonts

\title{Assignment 3: Report}
\author{Robson Edwards}
\date{November 22, 2018}							% Activate to display a given date or no date

\begin{document}

\maketitle
%\tableofcontents

I confirm that the following report and associated code is my own work, except where clearly indicated.

%If you fit a model to some data, consider whether there is a need for goodness-of-fit tests. State the assumptions of the methods you use, and consider how likely they are to be met – see the above advice on the Discussion section for more on this.

%Leave enough time to write a decent report. Good work badly reported will gain few marks. %LOL

\section*{Abstract}

We investigate poll data from the 2016 U.S. Election and attempt to answer the research question ``Do polls become more accurate closer to the election?'' To do this, we simulate large amounts of data [ from the estimated distribution of the real data ] / [using the properties of the real data]. 
We then estimate the correlation between poll date and poll accuracy using two different statistical tests. 

We find that... %TODO

We conclude that... %TODO

Our tests have a certain size and power... %TODO 

\section{Introduction}

``Poll results from US 2016 presidential elections aggregated from HuffPost Pollster, RealClearPolitics, polling firms and news reports.'' [TODO: cite dslabs]

%Give a concise background to the problem, referring to previous work in the field if that’s appropriate. For MT4613 you need to demonstrate that you understand the problem. Give a motivation for the work that’s going to be described in later sections – why did you do it (i.e., why is it scientifically worthwhile/interesting)? “Proper” scientific papers may also lay out the scientific hypotheses being tested – this might not be applicable to the reports you write for MT4613. Finish by describing in a couple of sentences how the rest of the report is structured (especially if you use a non-standard layout).

\section{Methods}

%Describe what you did, in enough detail that someone else could reproduce it. In academic papers this is sometimes not possible due to space constraints, so full details are relegated to an appendix; sometimes the computer code and input data are archived in an appendix (especially online appendices). For the reports in MT4613 you should be able to explain what you did concisely without the need for appendices. If the methods section is quite long, you may need subheadings to divide it into manageable parts. Don’t be afraid of using formulae here to explain your methods, but make sure you define all the symbols used. It may also be helpful to use a figure, for example to show a map of the study area, or a diagram explaining how various analyses fit together. Such figures will likely not be required in MT4613.

\section{Results}

Answer the research question 

%ay what you found. Results should be given in the same order as the work was laid out in the methods section, with the same (or nearly the same) subheadings as the methods (if you had any). Only report results that are relevant to the problem at hand, and were motivated by the methods section – don’t just dump any old output from a computer program into the results section. Summarize results into tables and figures as much as possible, and don’t repeat information given in tables in the text, just refer to it. For example “All five populations showed a statistically significant decline in numbers after the fire (Table 1).” See below for more advice on tables and figures. Consider how many significant figures/decimal places results should reasonably be reported to, and be consistent – there is often no need to report results to more than 2 or 3 decimal places.

%\section{Discussion}

%Assumptions in the methods and ...

%Actually Eiren doesn't want a discussion section. Oh well. 

%This section usually starts by discussing how reliable the results are. Lay out the assumptions of the methods used, consider how valid they are, and what the consequences of likely violations might be. After this, the discussion considers the implications of the results – what do they mean for the questions being addressed by the paper. This section often finishes with a brief consideration of what further work might be useful, often with specific suggestions.

\section{Conclusion}

%Concisely lay out the main conclusions of the paper. This section is often somewhat redundant, given that these are also given in the abstract (although in less detail). You’re probably better to finish the discussion with a pity [pithy?] concluding paragraph.

\section{References}

%This section is sometimes called “Literature cited”. Each printed text or web site referred to in the report should be listed here, in a standard format. The format can mimic any standard statistics journal; if in doubt, use the Harvard system of referencing (search on the web for information about this), which is something of a standard in science.

\section{Appendix}
%\appendix{} %which one?

\end{document}
