
\documentclass[10pt, oneside]{article}
\usepackage{geometry} 
\geometry{a4paper} 
\geometry{top = 1cm, left = 1.5cm, right = 1.5cm}
%\usepackage[parfill]{parskip}  % Activate to begin paragraphs with an empty line rather than an indent
\usepackage{graphicx}			% Use pdf, png, jpg, or eps§ with pdflatex; use eps in DVI mode
								% TeX will automatically convert eps --> pdf in pdflatex		
\usepackage{amssymb}

\usepackage{xcolor}

\pagenumbering{gobble}

%SetFonts

%SetFonts

\title{Assignment 3: Diagram}
\author{Robson Edwards}
\date{November 22, 2018}		% Activate to display a given date or no date

\newcommand \diabox[2]{\colorbox{lightgray}{\parbox{#1}{#2}}}

\begin{document}

\maketitle
%\tableofcontents

\begin{minipage}[t]{8cm}

\begin{minipage}[t]{8cm}

\section*{Problem}
\diabox{8cm}{
Research Question: ``Do polls become more accurate closer to the election?''

~

Assume poll errors are drawn from a normal distribution with a variance that is constant with respect to time.

\begin{itemize}
\item $H_0$: Poll error mean does not vary with respect to time.

\item $H_1$: Poll error is strictly decreasing with respect to time. 

\item Tests: Pearson and Spearman correlation
\end{itemize}

}
\end{minipage}

\begin{minipage}[t]{8cm}
~

\section*{Testing}
\diabox{8cm}{
So now we have four (one $H_0$ and three $H_1$) hypotheses, each of which corresponds to 19 simulation ``situations'' (4 showing different durations, 4 showing different pollsizes, 11 showing different grades) at three different sample sizes. 

~

For each of these $4 \times 19 \times 3 = 228$ situations we create 100 samples and then apply both statistical tests.
Then for the 57 $H_0$ situations we calculate size as the proportion of the 100 samples for which $H_0$ was incorrectly rejected.
For the 171 situations under various $H_1$'s we calculate power as the proportion of the time that $H_0$ was correctly rejected. 

~

Notably, for all of these, the output of interest is the proportion where the test rejected $H_0$. 
}

\end{minipage}

\end{minipage}
\hfill
\begin{minipage}[t]{1cm}

\vskip 2in

\Huge$\rightarrow$

\vskip 2in

$\leftarrow$\normalsize

\end{minipage}
\hfill
\begin{minipage}[t]{8cm}

\section*{Simulation}
\diabox{8cm}{
To determine \emph{size} we simulate \emph{situations} where $H_0$ is true. 

\begin{itemize}
\item Situation: Time has no effect on error so $\beta = 0$. In each of the below, draw errors from a normal with mean and sd as observed. Then attach a randomly selected enddate. Then report the simulated data

\begin{itemize}
\item consider different grades: A+, A, ... C-, D, and NA

\item consider different quartiles for duration

\item consider different quartiles for pollsize
\end{itemize}

\end{itemize}

To determine the \emph{power} we simulate in \emph{situations} where a $H_1$ is true. 

\begin{itemize}
\item Situation: Time has a tiny effect on error, $\beta = -0.001$. Now, draw errors from a normal with sd as observed, and a mean that decreases as a function of $\beta \times$enddate. 

\begin{itemize}
\item consider different subsets of the data as above
\end{itemize}

\item Situation: Time has a medium effect on error, $\beta = -0.01$. Process as above

\item Situation: Time has a large effect on error, $\beta = -0.1$. Process as above. 

\end{itemize}

For each of the above sub-situations we take normal random samples with the appropriate mean and sd, and with n = 100, 1000, 4000. 

Note in the above $\beta$ is a measure of relationship between error mean and time.

}
\end{minipage}

\end{document}  
